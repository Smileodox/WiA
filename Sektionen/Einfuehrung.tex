\section{Einleitung}
\subsection{Motivation und Bedeutung der Visualisierung von Unsicherheiten}
\cite{Padilla2021}
\subsection{Ziel der Arbeit und Forschungsfragen}

\section{Theoretische Grundlagen}
\subsection{Grundlagen der Risikowahrnehmung}
\cite{Larcher2020}
\subsection{Visualisierungstechniken und ihre psychologischen Effekte}
\cite{Joslyn2021}

\section{Einfluss von Visualisierungstechniken auf die Risikowahrnehmung}
\subsection{Unterschiedliche Visualisierungstechniken und ihre Eigenschaften}
\cite{Pang1997}
\cite{Kerr2023}
\cite{Brodlie2012ARO}
\cite{Haber1990}
\subsection{Studien zur Risikowahrnehmung in der Finanzanalyse}

\section{Verbesserung der Kommunikation durch Visualisierungstechniken}
\subsection{Kommunikation zwischen Finanzanalysten und Laieninvestoren}
\cite{Joslyn2021}
\subsection{Einsatz von Visualisierungen zur Darstellung von Unsicherheiten}

\section{Übertragung von Erkenntnissen aus anderen Domänen}
\subsection{Beispiele erfolgreicher Visualisierungstechniken aus anderen Bereichen}
\cite{Boller2010}
\subsection{Anwendbarkeit auf den Finanzsektor}

\section{Fazit und Ausblick}
\subsection{Zusammenfassung der wichtigsten Erkenntnisse}
\subsection{Mögliche zukünftige Forschungsrichtungen}

