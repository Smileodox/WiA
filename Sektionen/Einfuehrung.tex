\section{Einleitung}
Ein fundamentales Konzept in den Wirtschaftswissenschaften ist die von Frank H. Knight eingeführte Unterscheidung zwischen Risiko und 
Unsicherheit. Knight definierte Risiko als quantifizierbare Unsicherheit, also Situationen, in denen den möglichen Ergebnissen 
Wahrscheinlichkeiten zugeordnet werden können. Ein klassisches Beispiel hierfür sind Versicherungen oder Glücksspiele, bei denen 
aufgrund historischer Daten oder mathematischer Modelle die Wahrscheinlichkeiten der verschiedenen Ausgänge bekannt sind.

Unsicherheit hingegen, die Knight als "echte Unsicherheit" bezeichnet, bezieht sich auf Situationen, in denen diese Zuordnung nicht 
möglich ist. Dies tritt auf, wenn keine verlässlichen historischen Daten zur Verfügung stehen, um die Wahrscheinlichkeiten der 
verschiedenen Ergebnisse zu prognostizieren. Ein Beispiel hierfür wäre die Einführung eines innovativen Produkts auf den Markt, bei 
dem es keine vorherigen Daten gibt, die den Erfolg oder Misserfolg vorhersagen könnten. \cite{Knight1921}

Wenn man Unsicherheit aus der Perspektive der Datenvisualisierung betrachtet, stößt man auf viele ähnliche, jedoch auch zahlreiche 
unterschiedliche Definitionen. Haber und McNabb \cite{Haber1990} beschreiben einen generischen Prozess zur Visualisierung von Daten, 
der die verschiedenen Stufen von der Datenbeschaffung bis zur finalen Visualisierung abdeckt. Dieser Prozess, der auch von Pang et al. \cite{Pang1997} 
aufgegriffen wird, besteht aus mehreren Schritten, in denen Unsicherheit in unterschiedlichem Maße eingeführt und berücksichtigt
werden muss, und diese unterschiedlichen Definitionen verdeutlicht:

\begin{enumerate}
    \item \textbf{Datenbeschaffung}:
    In dieser Phase ist Unsicherheit inherent, sei es durch Messfehler/-ungenauigkeiten, die Beschaffung der Daten durch statistische Modelle oder unvollständige Daten.
    
    \item \textbf{Datenvorverarbeitung}:
    Die beschafften Daten müssen in einem nachgehenden Schritt aufbereitet werden, in welchem durch Interpolation von fehlenden Daten, ungenaue Transformationen oder Annahmen weitere Unsicherheit einfließen kann.
    
    \item \textbf{Datenverarbeitung und -analyse}:
    Diese Daten werden zur Visualisierung auf ein/mehrere geometrische Objekte gemappt. Dies stellt durch die verwendeten Algorithmen und Modelle eine neue Quelle von Unsicherheit dar.
    
    \item \textbf{Visualisierung}:
    Schließlich werden die Daten visualisiert. Hier können Unsicherheiten durch die Wahl der Visualisierungstechniken und Darstellungsparameter wie Farbskalen und Fehlerbalken an sich beeinflusst werden.
\end{enumerate}

\subsection{Die Bedeutung von Unsicherheit in Finanzmärkten}
Knight schrieb in seinem Werk ausserdem, dass in einem fairen Markt nur ein Unternehmer wirtschaftlich erfolgreich
sein kann, wenn dieser Unsicherheiten auf sich nimmt, da andernfalls jeder Marktteilnehmer die gleichen, korrekten 
Informationen hätte und somit kein Vorteil erarbeitet werden kann.
Wird diese aufzunehmende Unsicherheit aber im Entscheidungsprozess falsch oder unzureichend dargestellt, kann dies fatale Folgen nach sich 
ziehen. 

Man stelle sich beispielsweise die Prognose eines Aktienkurses anhand eines Monte-Carlo-Modells vor. Sollte ein Investor auf Basis 
dieser Prognose eine Entscheidung treffen, ist er einerseits mit der direkten quantitativen Unsicherheit der Vorhersage konfrontiert, 
also mit der Wahrscheinlichkeit, dass genau der gewählte Zweig der Simulation zutrifft, und der Varianz. Andererseits gibt es die 
indirekte qualitative Unsicherheit: Hat der Ersteller korrekt historische Daten verwendet? \cite{Padilla2021} Auf welcher Basis von Zufall wurde die Prognose erstellt –
 Volatilität oder fundierte ökonomische Kennzahlen des Unternehmens? Es gilt, diese Unsicherheiten so gut wie möglich darzustellen, 
 um den Investor bei seiner Entscheidungsfindung zu unterstützen und keine Trugschlüsse zuzulassen.

\subsection{Ziele der Arbeit}
Deshalb wird in dieser Arbeit untersucht, wie sich die Visualisierung von Unsicherheiten und Risiken auf Entscheidungsträger auswirkt, wie Techniken angewandt werden können um 
die Kommunikation von Information zwischen Laien und professionellen verbessert werden kann und wie bereits entwickelte Techniken aus anderen Domänen auf die Finanzwelt übertragen werden können.

Im folgendem Kapitel werden Grundlagen zur Entscheidungsfindungen von Menschen sowie Techniken der Visualisierung von Unsicherheiten beleuchtet,
daraufhin


\section{Theoretische Grundlagen}
Es wird die Basis geschaffen um Visualiserungstechniken sowie ihre Bedeutung in der Entscheidungsfindung besser verstehen zu können.

\subsection{Theorien zur Entscheidungsfindung unter Unsicherheit}
Im Folgendem werden Theorien zur Entscheidungsfindung unter Unsicherheit näher beleuchtet, welche laut Padille et al. notwendigerweiße berücksichtigt
werden müssen, um Visualisierungstechniken auf Basis ihrer Rolle in Entscheidungsfindungen in anbetracht von Unsicherheit bewerten zu können. \cite{VisualizationPsychology2023}

\subsubsection{Erwartungsnutzentheorie}

Die Erwartungsnutzentheorie (\ac{EUT}) ist ein grundlegendes Modell in der Entscheidungstheorie, das beschreibt, wie rationale Akteure Entscheidungen unter Unsicherheit 
treffen sollten. Entwickelt von John von Neumann und Oskar Morgenstern, basiert die EUT auf der Annahme, dass Individuen bei der 
Wahl zwischen unsicheren Alternativen jene Option bevorzugen, die den höchsten erwarteten Nutzen bietet. Der erwartete Nutzen eines 
Ergebnisses wird dabei als das Produkt aus der Wahrscheinlichkeit des Ergebnisses und dem subjektiven Wert (Nutzen) dieses Ergebnisses 
berechnet.

Die formale Darstellung der Erwartungsnutzentheorie lautet:

\begin{equation}
EU = \sum_{i=1}^{n} p_i \cdot u(x_i)
\end{equation}

Dabei ist \( EU \) der erwartete Nutzen, \( p_i \) die Wahrscheinlichkeit des Ergebnisses \( x_i \) und \( u(x_i) \) der Nutzen des 
Ergebnisses \( x_i \).

Die EUT geht davon aus, dass Individuen konsistente Präferenzen haben und stets die Alternative mit dem höchsten erwarteten Nutzen 
wählen. \cite{vonNeumann1944} Diese Theorie hat weitreichende Anwendungen in der Finanzwelt, beispielsweise bei der Analyse von 
Investitionsentscheidungen und Risiken. 

\subsubsection{Dual-Process-Theorie}

Die Dual-Process-Theorie beschreibt, wie Menschen zwei verschiedene Arten von Denkprozessen nutzen, um Entscheidungen zu treffen: 
intuitives (Type 1) und analytisches (Type 2) Denken. Fundierungen für diese Theorie wurden von Psychologen wie Daniel Kahneman und Keith 
Stanovich zum Beispie in \cite{Tversky74} über Jahre hinweg entwickelt und hebt hervor, dass Menschen je nach Situation und 
Komplexität der Aufgabe zwischen diesen beiden Denkmodi wechseln.

\begin{itemize}
    \item \textbf{Type 1 Prozesse}: Diese sind schnell, automatisch und erfordern wenig kognitive Anstrengung. 
    Sie basieren auf Intuition und Heuristiken, die oft aus Erfahrungen und erlernten Mustern resultieren. 
    Type 1 Prozesse sind nützlich in routinierten und vertrauten Situationen, können jedoch zu systematischen 
    Fehlern und Verzerrungen führen.
    \item \textbf{Type 2 Prozesse}: Diese sind langsam, bewusst und erfordern erhebliche kognitive Anstrengung. 
    Sie basieren auf logischem Denken und systematischer Analyse. Type 2 Prozesse kommen zum Einsatz, wenn komplexe 
    und neue Situationen eine gründliche Bewertung erfordern.
\end{itemize}

Die Dual-Process-Theorie erklärt, warum Menschen in vielen Situationen intuitive Entscheidungen treffen, die schnell und effizient 
sind, aber manchmal zu suboptimalen Ergebnissen führen. Sie betont auch die Notwendigkeit, analytisches Denken zu fördern, 
insbesondere bei komplexen und wichtigen Entscheidungen, bei denen Fehler schwerwiegende Konsequenzen haben können.


\subsection{Datenvisualisierung in der Finanzwelt}
\subsubsection{Visualisierungsprozess}
\subsubsection{Techniken zur Visualisierung von Unischerheiten}
\cite{Joslyn2021}

\section{Einfluss von Visualisierungstechniken auf die Risikowahrnehmung}
\subsection{Fallstudie Lotterie}
\cite{Larcher2020}
\cite{Pang1997}
\cite{Kerr2023}
\cite{Brodlie2012ARO}
\cite{Haber1990}
\subsection{Fallstudie Fantasy Football}

\section{Verbesserung der Kommunikation durch Visualisierungstechniken}
\subsection{Kommunikation zwischen Finanzanalysten und Laieninvestoren}
\cite{Joslyn2021}
\subsection{Einsatz von Visualisierungen zur Darstellung von Unsicherheiten}

\section{Übertragung von Erkenntnissen aus anderen Domänen}
\subsection{Beispiele erfolgreicher Visualisierungstechniken aus anderen Bereichen}
\cite{Boller2010}
\subsection{Anwendbarkeit auf den Finanzsektor}

\section{Fazit und Ausblick}
\subsection{Zusammenfassung der wichtigsten Erkenntnisse}
\subsection{Mögliche zukünftige Forschungsrichtungen}

