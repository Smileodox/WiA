\section{Einleitung}
Ein fundamentales Konzept in den Wirtschaftswissenschaften ist die von Frank H. Knight eingeführte Unterscheidung zwischen Risiko und 
Unsicherheit. Knight definierte Risiko als quantifizierbare Unsicherheit, also Situationen, in denen den möglichen Ergebnissen 
Wahrscheinlichkeiten zugeordnet werden können. Ein klassisches Beispiel hierfür sind Versicherungen oder Glücksspiele, bei denen 
aufgrund historischer Daten oder mathematischer Modelle die Wahrscheinlichkeiten der verschiedenen Ausgänge bekannt sind.

Unsicherheit hingegen, die Knight als "echte Unsicherheit" bezeichnet, bezieht sich auf Situationen, in denen diese Zuordnung nicht 
möglich ist. Dies tritt auf, wenn keine verlässlichen historischen Daten zur Verfügung stehen, um die Wahrscheinlichkeiten der 
verschiedenen Ergebnisse zu prognostizieren. Ein Beispiel hierfür wäre die Einführung eines innovativen Produkts auf den Markt, bei 
dem es keine vorherigen Daten gibt, die den Erfolg oder Misserfolg vorhersagen könnten. \cite{Knight1921}

Wenn man Unsicherheit aus der Perspektive der Datenvisualisierung betrachtet, stößt man auf viele ähnliche, jedoch auch zahlreiche 
unterschiedliche Definitionen. Haber und McNabb \cite{Haber1990} beschreiben einen generischen Prozess zur Visualisierung von Daten, 
der die verschiedenen Stufen von der Datenbeschaffung bis zur finalen Visualisierung abdeckt. Dieser Prozess, der auch von Pang et al. \cite{Pang1997} 
aufgegriffen wird, besteht aus mehreren Schritten, in denen Unsicherheit in unterschiedlichem Maße eingeführt und berücksichtigt
werden muss, und diese unterschiedlichen Definitionen verdeutlicht:

\begin{enumerate}
    \item \textbf{Datenbeschaffung}:
    In dieser Phase ist Unsicherheit inherent, sei es durch Messfehler/-ungenauigkeiten, die Beschaffung der Daten durch statistische Modelle oder unvollständige Daten.
    
    \item \textbf{Datenvorverarbeitung}:
    Die beschafften Daten müssen in einem nachgehenden Schritt aufbereitet werden, in welchem durch Interpolation von fehlenden Daten, ungenaue Transformationen oder Annahmen weitere Unsicherheit einfließen kann.
    
    \item \textbf{Datenverarbeitung und -analyse}:
    Diese Daten werden zur Visualisierung auf ein/mehrere geometrische Objekte gemappt. Dies stellt durch die verwendeten Algorithmen und Modelle eine neue Quelle von Unsicherheit dar.
    
    \item \textbf{Visualisierung}:
    Schließlich werden die Daten visualisiert. Hier können Unsicherheiten durch die Wahl der Visualisierungstechniken und Darstellungsparameter wie Farbskalen und Fehlerbalken an sich beeinflusst werden.
\end{enumerate}

\subsection{Die Bedeutung von Unsicherheit in Finanzmärkten}
Knight schrieb in seinem Werk ausserdem, dass in einem fairen Markt nur ein Unternehmer wirtschaftlich erfolgreich
sein kann, wenn dieser Unsicherheiten auf sich nimmt, da andernfalls jeder Marktteilnehmer die gleichen, korrekten 
Informationen hätte und somit kein Vorteil erarbeitet werden kann.
Wird diese aufzunehmende Unsicherheit aber im Entscheidungsprozess falsch oder unzureichend dargestellt, kann dies fatale Folgen nach sich 
ziehen. 

Man stelle sich beispielsweise die Prognose eines Aktienkurses anhand eines Monte-Carlo-Modells vor. Sollte ein Investor auf Basis 
dieser Prognose eine Entscheidung treffen, ist er einerseits mit der direkten quantitativen Unsicherheit der Vorhersage konfrontiert, 
also mit der Wahrscheinlichkeit, dass genau der gewählte Zweig der Simulation zutrifft, und der Varianz. Andererseits gibt es die 
indirekte qualitative Unsicherheit: Hat der Ersteller korrekt historische Daten verwendet? \cite{Padilla2021} Auf welcher Basis von Zufall wurde die Prognose erstellt –
 Volatilität oder fundierte ökonomische Kennzahlen des Unternehmens? Es gilt, diese Unsicherheiten so gut wie möglich darzustellen, 
 um den Investor bei seiner Entscheidungsfindung zu unterstützen und keine Trugschlüsse zuzulassen.
\subsection{Ziele der Arbeit}
Deshalb wird in dieser Arbeit untersucht, wie sich die Visualisierung von Unsicherheiten und Risiken auf Entscheidungsträger auswirkt, wie Techniken angewandt werden können um 
die Kommunikation von Information zwischen Laien und professionellen verbessert werden kann und wie bereits entwickelte Techniken aus anderen Domänen auf die Finanzwelt übertragen werden können.


\section{Theoretische Grundlagen}
\subsection{Grundlagen der Risikowahrnehmung}
\cite{Larcher2020}
\subsection{Visualisierungstechniken und ihre psychologischen Effekte}
\cite{Joslyn2021}

\section{Einfluss von Visualisierungstechniken auf die Risikowahrnehmung}
\subsection{Unterschiedliche Visualisierungstechniken und ihre Eigenschaften}
\cite{Pang1997}
\cite{Kerr2023}
\cite{Brodlie2012ARO}
\cite{Haber1990}
\subsection{Studien zur Risikowahrnehmung in der Finanzanalyse}

\section{Verbesserung der Kommunikation durch Visualisierungstechniken}
\subsection{Kommunikation zwischen Finanzanalysten und Laieninvestoren}
\cite{Joslyn2021}
\subsection{Einsatz von Visualisierungen zur Darstellung von Unsicherheiten}

\section{Übertragung von Erkenntnissen aus anderen Domänen}
\subsection{Beispiele erfolgreicher Visualisierungstechniken aus anderen Bereichen}
\cite{Boller2010}
\subsection{Anwendbarkeit auf den Finanzsektor}

\section{Fazit und Ausblick}
\subsection{Zusammenfassung der wichtigsten Erkenntnisse}
\subsection{Mögliche zukünftige Forschungsrichtungen}

